\label{sec:appendix-theory}

This appendix contains the proof for \lemmaref{lem:subopt-convex}.
We first present some supporting material.

\subsection{Supporting material}
We begin by recalling the definition and properties of quasiconvex functions on $\R$.
We then recall some basic facts about scalar LQR.

\begin{definition}
	A function $f : \cD \mapsto \R\ $ on the convex domain $\ \cD \subseteq \R^n$
	is \emph{quasiconvex} if its sublevel sets
	\(
		\cD_\alpha = \{ x \in \cD : f(x) \leq \alpha \}
	\)
	are convex for all $\alpha \in \R$.
\end{definition}

\begin{lemma}[\citet{boyd-convex}, \S 3.4]
\label{lem:quasiconvex}
	The following facts hold for quasiconvex functions on a convex $\cD \subseteq \R$:
	\begin{enumerate}[label=(\alph*)]
		\item \label{alternatives}
			If $f: \R \mapsto \R$ is continuous,
			then $f$ is quasiconvex
			if and only if at least one of the following conditions holds on $\cD$:
			\begin{enumerate}[label=\arabic*.]
				\item $f$ is nondecreasing.
				\item $f$ is nonincreasing.
				\item \label{alternative-incdec}
					There exists $c \in \cD$ such that
					for all $t \in \cD$,
					if $t < c$ then $f$ is nonincreasing,
					and if $t \geq c$ then $f$ is nondecreasing.
			\end{enumerate}
		\item \label{secondorder}
			If $f : \R \mapsto \R$ is twice differentiable
			and
			$\partial^2 f / \partial x^2 > 0$
			for all $x \in \cD$ where
			$\partial f / \partial x = 0$,
			then $f$ is quasiconvex on $\cD$.
		\item \label{quotient}
			If $f(x) = p(x) / q(x)$,
			where $p: \R \mapsto \R$ is convex with $p(x) \geq 0$ on $\cD$
			and $q: \R \mapsto \R$ is concave with $q(x) > 0$ on $\cD$,
			then $f$ is quasiconvex on $\cD$.
	\end{enumerate}
\end{lemma}

\noindent The following facts about scalar LQR problems can be derived from
the LQR Riccati equation and some calculus (not shown).
\begin{lemma}
\label{lem:scalar-lqr-facts}
For the scalar LQR problem with $a > 0, b > 0$ and $q = r = 1$,
the optimal linear controller $\kopt{a,b}$ is given by the closed-form expression
	\[
		k^\star_{a,b}
		=
		\min_{k \in \R} J_{a,b}(k)
		=
		\min_{k \in \R} \frac{1 + k^2}{-2(a + bk)}
		=
		-\frac{a + \sqrt{a^2 + b^2}}{b}.
	\]
For fixed $a$, the map from $b$ to $k^\star_{a,b}$
is continuous and strictly increasing on the domain $b \in (0, \infty)$
and has the range $(-\infty, -1)$.
For any $k \in (-\infty, -1)$,
there exists a unique $b_k \in (-\infty, -1)$
for which $k = k^\star_{a,b_k}$, given by
\[
	b_k = \frac{2ak}{1 - k^2}.
\]
\end{lemma}

\newcommand{\bkdomain}{\cD}
\newcommand{\bdomain}{\cD^b}
\newcommand{\kdomain}{\cD^k}

\subsection{Proof of \lemmaref{lem:subopt-convex}}

We first recall the statement of the lemma.
\quasi*

\noindent Instead of a monolithic proof,
we present supporting material in \lemmaref{lem:ratio-quasi,lem:cost-quasi}.
We then show the main result in \lemmaref{lem:increasing-main},
which considers $\alpha$-suboptimal neighborhoods on all of $\R$
instead of restricted to $\Bset$.
\lemmaref{lem:subopt-convex} will follow as a corollary.

We proceed with more setup.
Recall that the scalar \DDFproblem\ is defined by
$\Aset = \{a\}$ and $\Bset = [\recip \theta, 1]$, where $a > 0$.
For this section,
let
\[
	\bkdomain
	= \{ (b, k) \in (0, \infty) \times \R : a + bk < 0 \}
\]
(note that $J_b(k) < \infty \iff a + bk < 0$).
Denote its projections by
$\bdomain(k) = \{ b: (b, k) \in \bkdomain \}$
and
$\kdomain(b) = \{ k: (b, k) \in \bkdomain \}$.
We compute the suboptimality ratio $r : \bkdomain \mapsto \R$ by
\[
	r(b, k) =
	\frac{J_b(k)}{J^\star_b}
	=
	\nofrac{
		\frac{
			1 + k^2
		}{
			-2(a + bk)
		}
	}{
		\frac{
			a + \sqrt{a^2 + b^2}
		}{
			b^2
		}
	}
	=
	\frac{
		(1 + k^2) b^2
	}{
		-2(a + bk)(a + \sqrt{a^2 + b^2})
	}.
\]
We denote its sublevel sets with respect to $b$ for fixed $k$ by
\[
	\bdomain_\alpha(k) = \{ b \in \bdomain(k) : r(b, k) \leq \alpha \}.
\]
\begin{lemma}
\label{lem:ratio-quasi}
	For fixed $k < 0$, the ratio $r(b, k)$ is quasiconvex on $\bdomain_k$,
	and there is at most one $b \in \bdomain_k$ at which $\partial r / \partial b = 0$.
\end{lemma}
\begin{proof}
	By inspection, $r(b, k)$ is smooth on $\bdomain$.
	We now show that the second-order condition of \lemmaref{lem:quasiconvex}\ref{secondorder} holds.
	\input{ratio_intervals_proof.tex}
	The conclusion follows from \lemmaref{lem:quasiconvex}\ref{secondorder}.
\end{proof}

\begin{lemma}
\label{lem:cost-quasi}
	For fixed $b$, the cost $J_b(k)$ is quasiconvex on $\kdomain(b)$.
	Also, $J_b(k)$ is not monotonic,
	so case \ref{alternative-incdec} of \lemmaref{lem:quasiconvex}\ref{alternatives} applies.
\end{lemma}
\begin{proof}
	We have
	\[
		J_b(k)
		=
		\frac{
			1 + k^2
		}{
			-2(a + bk)
		}.
	\]
	The numerator is nonnegative and convex on $k \in \R$.
	The denominator is linear (hence concave) and positive on $\kdomain(b)$.
	Quasiconvexity follows from \lemmaref{lem:quasiconvex}\ref{quotient}.
	Nonmonotonicity follows from the fact that $J_b(k)$ is smooth on $\kdomain(b)$
	and has a unique optimum at $\kopt{b}$, which is not on the boundary of $\kdomain(b)$.
\end{proof}

\noindent We now combine these into the main result.

\begin{lemma}
\label{lem:increasing-main}
	For a scalar \DDFproblem,
	if $\alpha > 1$ and $k < -1$,
	then $\bdomain_\alpha(k)$ is either:
	a bounded closed interval $[b_1, b_2]$,
		with $b_1$ and $b_2$ increasing in $k$, or
	a half-bounded closed interval $[b_1, \infty)$,
		with $b_1$ increasing in $k$.
\end{lemma}

\begin{proof}
	By \lemmaref{lem:ratio-quasi}, due to quasiconvexity $\bdomain_\alpha$ is convex.
	The only convex sets on $\R$ are the empty set and all types of intervals:
	open, closed, and half-open.
	We know $\bdomain_\alpha$ is not empty because it contains $b_k$.
	We can further assert that
	$\bdomain_\alpha$ has a closed lower bound
	because $\lim_{b \to (-a/k)} r(b, k) = \infty$
	(see \citet[\S A.3.3]{boyd-convex} for details).
	However, the upper bound may be closed or infinite.
	We handle the two cases separately.

	\paragraph{Bounded case.}
	Fix $k_0 < -1$.
	Suppose $\bdomain_\alpha(k_0) = [b_1, b_2]$ for $0 < b_1 < b_2 < \infty$.
	By the implicit function theorem (IFT), at any $(b_0, k_0)$ satisfying $r(b_0, k_0) = \alpha$,
	if $\partial r / \partial b|_{b_0, k_0} \neq 0$
	then there exists an open neighborhood around $(b_0, k_0)$
	for which the solution to $r(b, k) = \alpha$ can be expressed as
	$(g(k), k)$, where $g$ is a continuous function of $k$ and
	\[
		\left. \frac{\partial g(k)}{\partial k} \right|_{k_0} =
		\left. - \left( \frac{\partial r}{\partial b} \right)^{-1}
		\frac{\partial r}{\partial k} \right|_{b_0, k_0}.
	\]
	By the continuity and quasiconvexity of $r$,
	and the fact that $\partial r / \partial b = 0$ only at $b_k$
	(\lemmaref{lem:ratio-quasi})
	we know that $r(b_1, k_0) = r(b_2, k_0) = \alpha$
	and
	\[
		\left. \frac{\partial r}{\partial b} \right|_{b_1, k_0} < 0
		\quad \text{and} \quad
		\left. \frac{\partial r}{\partial b} \right|_{b_2, k_0} > 0.
	\]
	By \lemmaref{lem:scalar-lqr-facts},
	since $k < -1$
	there exists $b_k > 0$ satisfying
	$k = k^\star_{b_k}$.
	Since $r(b_k, k) = 1$ and $\alpha > 1$, we know $b_{k_0} \in (b_1, b_2)$.
	Again by \lemmaref{lem:scalar-lqr-facts},
	the map from $b$ to $k^\star_b$ is increasing in $b$.
	Therefore,
	\(
		{k^\star_{b_1} < k_0 < k^\star_{b_2}}.
	\)
	By the quasiconvexity and nonmonotonicity of $J_b(k)$ from \lemmaref{lem:cost-quasi},
	via \lemmaref{lem:quasiconvex}\ref{alternatives}
	we have
	\[
		\left. \frac{\partial r}{\partial k} \right|_{b_1, k_0} \geq 0
		\quad \text{and} \quad
		\left. \frac{\partial r}{\partial k} \right|_{b_2, k_0} \leq 0.
	\]
	Therefore, the functions $g_1, g_2$ satisfying the conclusion of the IFT
	in the neighborhoods around $(b_1, k_0)$ and $(b_2, k_0)$ respectively
	also satisfy
	\[
		\left. \frac{ \partial g_1(k)}{ \partial k } \right|_{b_1, k_0} \geq 0
		\quad \text{and} \quad
		\left. \frac{ \partial g_2(k)}{ \partial k } \right|_{b_2, k_0} \geq 0.
	\]
	Therefore, $b_1$ and $b_2$ are locally nondecreasing in $k$.
	

	\paragraph{Unbounded case.}
	Suppose $\bdomain_\alpha(k) = [b_1, \infty)$ for $b_1 < \infty$.
	By the same IFT argument as in the bounded case, $b_1$ is increasing in $k$.
	By the quasiconvexity of $r$ in $b$,
	the value of $r$ is increasing for $b  > b_k$,
	but the definition of $\bdomain_\alpha(k)$ implies that
	$r(b, k) \leq \alpha$ for all $b > b_k$.
	Therefore, $\lim_{b \to \infty} r(b, k)$ exists
	and is bounded by $\alpha$.
	In particular,
	\begin{equation*}
		\begin{split}
			\lim_{b \to \infty} r(b, k)
			&=
			\lim_{b \to \infty}
			\frac{
				(1 + k^2) b^2
			}{
				-2(a + bk)(a + \sqrt{a^2 + b^2})
			}
			\\ &=
			\lim_{b \to \infty}
			\frac{
				(1 + k^2) b^2 / b^2
			}{
				-2(a + bk)(a + \sqrt{a^2 + b^2}) / b^2
			}
			\\ &=
			-
			\frac{
				1 + k^2
			}{
				2k
			}.
		\end{split}
	\end{equation*}
	Taking the derivative shows that this value is decreasing in $k$ for $k < 0$.
	Therefore, if $k < k' < 0$ then
	\[
		\lim_{b \to \infty} r(b, k') \leq \lim_{b \to \infty} r(b, k) \leq \alpha.
	\]
	The property that $r(b, k')$ is increasing in $b$ for $b > b_k$
	further ensures that
	$r(b, k') \leq \alpha$ for all $b > b_k$.
	Therefore, $\bdomain_\alpha(k')$ is also unbounded.
\end{proof}

\noindent For completeness, we prove \lemmaref{lem:subopt-convex}.

\quad

\begin{proof} (of \lemmaref{lem:subopt-convex}).
	By \lemmaref{lem:increasing-main},
	$\bdomain_\alpha(k)$ is either
	a bounded closed interval $[b_1, b_2]$,
		with $b_1$ and $b_2$ increasing in $k$, or
	a half-bounded closed interval $[b_1, \infty)$,
		with $b_1$ increasing in $k$.
	Recall that $\suboptneighb{\alpha}{k} = \bdomain_\alpha(k) \cap \Bset$
	with $\Bset = [\recip \theta, 1]$.
	Therefore, the half-bounded case can be reduced to the bounded case with $b_2 = 1$.
	The intersection can be expressed as
	\[
		\suboptneighb{\alpha}{k} = [\max\{b_1, \textstyle \recip \theta\}, \min\{b_2, 1\}],
	\]
	where the interval $[a, b]$ is defined as the empty set if $a > b$.
	Taking the maximum or minimum of a nonstrict monotonic function and a constant
	preserves the monotonicity, so we are done.
\end{proof}
