
\label{sec:quadrotor}
As an example of a realistic \DDFproblem,
we consider the quadrotor helicopter illustrated in \Cref{fig:quadrotor}.
Near the hover state, its full nonlinear dynamics are well approximated by a linearization.
The state is given by
$
	x = (\pos, \vel, \attitude, \anglevel),
$
where $\pos \in \R^3$ is position,
$\vel \in \R^3$ is linear velocity,
$\attitude \in \R^3$ is attitude Euler angles,
and $\anglevel \in \R^3$ is angular velocity.
The inputs
$u \in \R^4_{\geq 0}$
are the squared angular velocities of the propellers.

Many factors influence the response to inputs,
including geometry, mass, moments of inertia, motor properties, and propeller aerodynamics.
These can be combined and partially nondimensionalized
into four control authority parameters to form $\env \in \Envs$.
The hover state occurs at $x = \zero,\ u \propto \one$,
where the constant input counteracts gravity.
The linearized dynamics are given by
\newcommand{\Gmatrix}{\begin{bsmallmatrix*}[r] 0 & g & 0 \\ \hspace{-1mm}-g & 0 & 0 \\ 0 & 0 & 0 \\ \end{bsmallmatrix*}}
\begin{equation*}
\label{eq:quad-linearized}
	\dot x
	=
	\underbrace{
		\begin{bsmallmatrix}
			\zero &     I & \zero    & \zero \\
			\zero & \zero & G & \zero \\
			\zero & \zero & \zero    &     I \\
			\zero & \zero & \zero    & \zero \\
		\end{bsmallmatrix}
	}_A
	x
	+
	\underbrace{
		\begin{bsmallmatrix}
			\zero    & \zero \\
			\hat e_z & \zero \\
			\zero    & \zero \\
			\zero    &     I \\
		\end{bsmallmatrix}
	}_U
	\underbrace{
		\begin{bsmallmatrix}
			\sigma_z & & & \\
			& \sigma_\roll & & \\
			& & \sigma_\pitch & \\
			& & & \sigma_\yaw \\
		\end{bsmallmatrix}
	}_\Sigma
	\underbrace{
		\begin{bsmallmatrix*}[r]
			 1 &  1 &  1 &  1 \\
			 1 & -1 & -1 &  1 \\
			-1 & -1 &  1 &  1 \\
			 1 & -1 &  1 & -1 \\
		\end{bsmallmatrix*}
	}_{V^\top}
	u,
	\quad
	G = \Gmatrix,
\end{equation*}
where $g$ is the gravitational constant
and $\hat e_z = [0\ 0\ 1]^\top$.
The parameters
$(\sigma_z,\ \sigma_\roll,\ \sigma_\pitch,\ \sigma_\yaw)$
denote
the thrust, roll, pitch, and yaw authority constants respectively.
Since we use the convention $\sigma \in [\recip \theta, 1]$,
the maximum value of each constant can be varied
by scaling the columns of $U$.




\begin{figure}[tpb]
	\begin{minipage}{0.28\textwidth}
		\tikzstyle{axis} = [-latex, thick]
\tikzstyle{thrust} = [-latex]
\tikzstyle{rot} = [-latex]
\tikzstyle{propdisc} = [fill=gray!20, fill opacity=0.90]
\tikzstyle{propmotion} = [gray!80]
\tikzstyle{propcw} = [->,black!75]
\tikzstyle{propccw} = [<-,black!75]

\tdplotsetmaincoords{45}{160}
\begin{tikzpicture}[scale=0.70,tdplot_main_coords]

  \pgfmathsetmacro{\xyaxlen}{2.5}
  \pgfmathsetmacro{\zaxlen}{2.7}
  \draw[axis] (0,0,0) -- (\xyaxlen,0,0) node[anchor=north east]{$x$};
  \draw[axis] (0,0,0) -- (0,\xyaxlen,0) node[anchor=west]{$y$};


  \tdplotsetrotatedcoords{0}{0}{45}
  \begin{scope}[tdplot_rotated_coords]
    \pgfmathsetmacro{\thick}{0.13}
    \pgfmathsetmacro{\shaftctr}{1.5}
    \pgfmathsetmacro{\len}{\shaftctr+\thick}
    \pgfmathsetmacro{\z}{0.0}
    \draw[fill=gray!70,tdplot_rotated_coords]
      (-\len,\thick,\z) --
      (-\thick,\thick,\z) --
      (-\thick,\len,\z) --
      (\thick,\len,\z) --
      (\thick,\thick,\z) --
      (\len,\thick,\z) --
      (\len,-\thick,\z) --
      (\thick,-\thick,\z) --
      (\thick,-\len,\z) --
      (-\thick,-\len,\z) --
      (-\thick,-\thick,\z) --
      (-\len,-\thick,\z) --
      cycle;

    \newcommand{\prop}[3]{
      \pgfmathsetmacro{\propradius}{0.8}
      \draw[propdisc] (#1,#2,0) circle (\propradius);
      \tdplotdefinepoints(#1,#2,0)(#1-0.15,#2+0.2,0)(#1+0.3,#2+0.2,0)
      \tdplotdrawpolytopearc[#3]{0.4*\propradius}{}{}
    }
    \prop{\shaftctr}{0}{propccw}
    \prop{0}{\shaftctr}{propcw}
    \prop{-\shaftctr}{0}{propccw}
    \prop{0}{-\shaftctr}{propcw}

    \pgfmathsetmacro{\thrusttext}{0.05}
    \pgfmathsetmacro{\thrustlen}{1.15}
    \draw[thrust] ( \shaftctr,0,0) -- ++(0,0,\thrustlen) node[pos=\thrusttext, left] {$u_1$};
    \draw[thrust] (0, \shaftctr,0) -- ++(0,0,\thrustlen) node[pos=\thrusttext,right] {$u_2$};
    \draw[thrust] (-\shaftctr,0,0) -- ++(0,0,\thrustlen) node[pos=\thrusttext,right] {$u_3$};
    \draw[thrust] (0,-\shaftctr,0) -- ++(0,0,\thrustlen) node[pos=\thrusttext, left] {$u_4$};
  \end{scope}

  \draw[axis] (0,0,0) -- (0,0,\zaxlen) node[anchor=south]{$z$};

  \pgfmathsetmacro{\arcinset}{0.4}
  \pgfmathsetmacro{\arcradius}{0.4}
  \tdplotdefinepoints(\xyaxlen-\arcinset,0,0)(\xyaxlen-\arcinset,0.1,-1)(\xyaxlen-\arcinset,0,1)
  \tdplotdrawpolytopearc[rot]{\arcradius}{anchor=north}{$\phi$}
  \tdplotdefinepoints(0,\xyaxlen-\arcinset,0)(-0.1,\xyaxlen-\arcinset,0.3)(0.1,\xyaxlen-\arcinset,-.2)
  \tdplotdrawpolytopearc[rot]{\arcradius}{anchor=north east}{$\theta$}
  \tdplotdefinepoints(0,0,\zaxlen-\arcinset)(0,-1,\zaxlen-\arcinset)(1,1.0,\zaxlen-\arcinset)
  \tdplotdrawpolytopearc[rot]{\arcradius}{anchor=east}{$\psi$}

\end{tikzpicture}

	\end{minipage}
	\hfill
	\begin{minipage}{0.68\textwidth}
		\caption{%
			Quadrotor helicopter with position states $x, y, z$,
			attitude states $\roll, \pitch, \yaw$,
			and propeller speed inputs $u_1, u_2, u_3, u_4$.
			The linearized dynamics at hover,
			subject to variations in mass, geometry, etc.,
			can be expressed in decomposed dynamics form---%
			see \Cref{sec:quadrotor}.
		}%
		\label{fig:quadrotor}
	\end{minipage}
\end{figure}
