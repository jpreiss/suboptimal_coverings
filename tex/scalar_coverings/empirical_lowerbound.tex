\label{sec:matrix-lb-ideas}

We now present intuition-building experiments towards a covering number lower bound
for matrix \DDFproblems.
A lower bound requires
a class of \DDFproblem\
that can be instantiated for any dimensionality $d$.
Two ``extremal'' systems come to mind:
\emph{minimum coupling},
where $A = \textstyle I$,
and \emph{maximum coupling},
where $A = \textstyle \recip n \one$.
Note that for minimum coupling,
an $\alpha$-suboptimal policy
is not necessarily $\alpha$-suboptimal on each scalar subsystem---%
if it were, the lower bound $\log (\theta)^d$ would trivially follow
from the results in \Cref{sec:results-scalar}.




\label{sec:2d-visualization}

\begin{figure}[t]
	\(
		\hspace{7mm}
		\overbrace{\hspace{65mm}}^{A \mathop{=} I}
		\hspace{10mm}
		\overbrace{\hspace{65mm}}^{A \mathop{=} \recip n \one}
	\)
	\adjustbox{clip,trim=4.5mm 1mm 0 4.5mm}{
		\inputpgf{figures}{neighborhoods_2x2.pgf}
	}
	\vspace{-2mm}
	\caption{
		$\alpha$-suboptimal neighborhoods
		for geometric grid partition in 2D system.
		\emph{Left:}
		minimum coupling; $A = I$.
		\emph{Right:}
		maximum coupling; $A = \recip n \one$.
		\emph{Columns:} varying suboptimality threshold $\alpha$.
		All axes are logarithmic. %
		Colors have no meaning.
		Discussion in \Cref{sec:2d-visualization}.
	}
	\label{fig:neighborhoods-2x2}
\end{figure}


We show approximate suboptimal neighborhoods
for a two-dimensional system in \Cref{fig:neighborhoods-2x2}.
We select a geometric grid of $\Sigma$ values (indicated by the circular markers)
and synthesize their LQR-optimal controllers.
Then, we evaluate the suboptimality ratio of each controller on a finer grid of $\Sigma$ values
to get approximate neighborhoods, indicated by the semi-transparent regions.
We repeat this experiment
with three values of $\alpha$
for both choices of $A$.

Interestingly, the neighborhoods for $A=I$ are not always connected.
In the plot for ${\alpha = 1.05}$ (far left),
the neighborhood for the minimal $\Sigma$
has another component that overlaps other neighborhoods
to its top and right.
If we increase to $\alpha = 1.1$,
the components join
into an ``L''-shaped region.
In contrast, the neighborhoods for $A = \recip n \one$ seem more well-behaved.
For both choices of $A$,
the neighborhoods are of comparable size.


To verify that this behavior %
is not an artifact of the two-dimensional case only,
we repeat the experiment in three dimensions.
\Cref{fig:neighborhoods-3x3} shows neighborhoods
of one controller $K = \Kopt{(2/\theta) I}$
for $\alpha$ ranging from $1.04$ to $1.2$.
As $\alpha$ grows, $\suboptneighb{\alpha}{K}$ shows similar topological phases as the $2$D case.
In the simply-connected phase (large $\alpha$),
the neighborhood appears to include
any $\Sigma$ where at least one $\sigma_i$ is sufficiently small.
If this property holds in higher dimensions,
then it would be possible
to construct a cover using only controllers of uniform gain in all dimensions
for large $\alpha$.

\begin{figure}[t]
	\hfill
	\foreach \alpha in {1.0367,1.0526,1.0852,1.1137,1.2000}
	{
		\includegraphics[width=0.17\textwidth,trim={60px 130px 80px 160px},clip]{neighborhoods_3x3/alpha_\alpha.png}
		\hfill
	}
	\caption{%
		$\alpha$-suboptimal neighborhoods for the three-dimensional decomposed dynamics system with
		minimal coupling
		($A = U = V^\top = I_{3 \times 3}$)
		and breadth $\breadth = 100$.
		Neighborhoods shown for
		$\alpha$ ranging from $1.04$ to $1.2$
		with a fixed
		controller.
	}
	\label{fig:neighborhoods-3x3}
\end{figure}


